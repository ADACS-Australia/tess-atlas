\documentclass[modern]{aastex63}

% \pdfoutput=1

% %\usepackage{lmodern}
\usepackage{microtype}
% \usepackage{url}
% \usepackage{amsmath}
% \usepackage{amssymb}
% \usepackage{natbib}
% \usepackage{multirow}
% \usepackage{graphicx}
% \bibliographystyle{aasjournal}

% \usepackage{mathtools}
% \usepackage{calc}
% \usepackage{etoolbox}
\usepackage{xspace}
\usepackage{xcolor}
% \usepackage[T1]{fontenc} % https://tex.stackexchange.com/a/166791
% \usepackage{textcomp}
\usepackage{ifxetex}
\ifxetex
\usepackage{fontspec}
\defaultfontfeatures{Extension = .otf}
\fi
\usepackage{fontawesome}


% references to text content
\newcommand{\documentname}{\textsl{Article}}
\newcommand{\figureref}[1]{\ref{fig:#1}}
\newcommand{\Figure}[1]{Figure~\figureref{#1}}
\newcommand{\figurelabel}[1]{\label{fig:#1}}
\newcommand{\eqref}[1]{\ref{eq:#1}}
\newcommand{\Eq}[1]{Equation~(\eqref{#1})}
\newcommand{\eq}[1]{\Eq{#1}}
\newcommand{\eqalt}[1]{Equation~\eqref{#1}}
\newcommand{\eqlabel}[1]{\label{eq:#1}}

% TODOs
\newcommand{\todo}[3]{{\color{#2}\emph{#1}: #3}}
\newcommand{\dfmtodo}[1]{\todo{DFM}{red}{#1}}
\newcommand{\avi}[1]{\todo{Avi}{orange}{#1}}
\newcommand{\alltodo}[1]{\todo{TEAM}{red}{#1}}
\newcommand{\citeme}{{\color{red}(citation needed)}}


% % typography obsessions
% \setlength{\parindent}{3.0ex}

% % from: https://github.com/rodluger/corTeX
% % Add code, proof, and animation hyperlinks
% \definecolor{linkcolor}{rgb}{0.1216,0.4667,0.7059}
% \newcommand{\codeicon}{{\color{linkcolor}\faFileCodeO}}
% \newcommand{\prooficon}{{\color{linkcolor}\faPencilSquareO}}
% \input{gitlinks}

% % Define a proof environment for open source equation proofs
% \newtagform{eqtag}[]{(}{)}
% \newcommand{\currentlabel}{None}
% \newenvironment{proof}[1]{%
% \ifstrempty{#1}{%
% \renewtagform{eqtag}[]{\raisebox{-0.1em}{{\color{red}\faPencilSquareO}}\,(}{)}%
% }{%
% \renewtagform{eqtag}[]{\prooflink{#1}\,(}{)}%
% }%
% \usetagform{eqtag}%
% \renewcommand{\currentlabel}{#1}
% \align%
% }{%
% \endalign%
% \renewtagform{eqtag}[]{(}{)}%
% \usetagform{eqtag}%
% \message{<<<\currentlabel: \theequation>>>}%
% }

% % Define the `oscaption` command for open source figure captions
% \newcommand{\oscaption}[2]{\caption{#2 \codelink{#1}}}

% Projects:
\newcommand{\project}[1]{\textsf{#1}}

\newcommand{\python}{\project{Python}}
\newcommand{\cython}{\project{Cython}}
\newcommand{\cpp}{\project{C++}}
\newcommand{\jupyter}{\project{Jupyter}}

\newcommand{\exoplanet}{\project{exoplanet}}
\newcommand{\lightkurve}{\project{lightkurve}}
\newcommand{\starry}{\project{starry}}
\newcommand{\radvel}{\project{RadVel}}
\newcommand{\batman}{\project{batman}}
\newcommand{\theano}{\project{Theano}}
\newcommand{\pymc}{\project{PyMC3}}
\newcommand{\celerite}{\project{celerite}}
\newcommand{\isochrones}{\project{isochrones}}
\newcommand{\dynesty}{\project{dynesty}}
\newcommand{\astroquery}{\project{astroquery}}
\newcommand{\jupytetr}{\project{astroquery}}

\newcommand{\tess}{\project{TESS}}
\newcommand{\kepler}{\project{Kepler}}
\newcommand{\gaia}{\project{Gaia}}

\newcommand{\cks}{\project{CKS}}
\newcommand{\mesa}{\project{MESA}}
\newcommand{\mast}{\project{MAST}}
\newcommand{\mist}{\project{MIST}}
\newcommand{\exofop}{\project{ExoFOP}}

% math
\newcommand{\T}{\ensuremath{\mathrm{T}}}
\newcommand{\dd}{\ensuremath{ \mathrm{d}}}
\newcommand{\unit}[1]{{\ensuremath{ \mathrm{#1}}}}
\newcommand{\bvec}[1]{{\ensuremath{\boldsymbol{#1}}}}





\shorttitle{The \tess\ Atlas}
\shortauthors{the authors}

\begin{document}

\title{The \tess\ Atlas}



\author{Author list TBD}
% \correspondingauthor{Daniel Foreman-Mackey}
\email{foreman.mackey@gmail.com}

\author[0000-0002-4146-1132{Avi Vajpeyi}
\affiliation{
    School of Physics and Astronomy,
    Monash University,
    Clayton VIC 3800,
    Australia
}
\affiliation{
OzGrav: The ARC Centre of Excellence for Gravitational Wave Discovery,
Clayton VIC 3800,
Australia
}

\author[0000-0002-9328-5652]{Daniel Foreman-Mackey}
\affiliation{
    Center for Computational Astrophysics,
    Flatiron Institute,
    162 5th Ave,
    New York, NY 10010
}




\begin{abstract}

This is the \tess\ Atlas.
\tess\ discovered N exoplanet candidates in N years of data.
We provide revised transit parameters and accompanying posterior distributions for all the \tess\ objects of interest.

\end{abstract}

% \keywords{editorials, notices ---
% miscellaneous --- catalogs --- surveys}

\section{Introduction} \label{sec:intro}

In this work we present the uniform modeling of all TESS object of interest (TOIs) utilizing Bayesian framework that provides robust estimates of the uncertainities for all of the planet parameters.

Section~\ref{sec:prob-model} describes our transit lightcurve model and the Bayesian framework we use to estimate parameters of exoplanet systems from the observed data.

\section{Probabilistic Model} \label{sec:prob-model}

\subsection{Transit lightcurve model}
We assume that exoplanets are on circular, non-interacting Keplerian orbits around their host star.
Additonally, we approximate the host star's stellar limb darkening profile using \citet{Kipping:2013}'s quadratic limb darkening law.
Finally, we compute the exoplanet's resultant quadratic limb-darkened transit lightcurve using an analytical model implemented in \starry.
The stellar variables in this model are parameterised by
the baseline relative flux of the light curve $f0$,
the mean stellar density $\rho_\star$,
and two parameters describing the quadratic limb-darkening profile of the
star $u_1, u_2$.
Each of the $n$ exoplanets in the system are parameterised by the planets'
orbital period $P_{n}$,
transit duration $T_{n}$,
phase or epoch $t0_n$,
impact parameter $b_n$,
and the radius of the planet $r_n$ in units of the stellar radius $R_\star$.

We use \exoplanet\, \lightkurve


In the case when there are multiple transits present in the data, we use a second reference time $tmax$ instead of $P$



\subsection{Bayesian Framework}


Likelihood, GP, Priors, \celerite


Table with priors

%
%
%GPs are stochastic models consisting of a mean function
%$\mu_\vec{\theta}(\vec{x})$ and a covariance, autocorrelation, or ``kernel''
%function $K$ parameterized by
%$\vec{\theta}$ and $\vec{\alpha}$ respectively.
%Under this model, the log-likelihood function $\mathcal{L}
%(\vec{\theta},\,\vec{\alpha})$ with a dataset
%
%\begin{align}\eqlabel{gp-likelihood}
%\ln \mathcal{L} (\vec{\theta},\,\vec{\alpha})
%&= -\frac{1}{2} r(\vec{\theta})^{T}\, K(\vec{\alpha})^{-1}\, r(\vec{\theta}) \nonumber \\
%   -\frac{1}{2} \log \det K(\vec{\alpha})
%
%\end{align}





\section{Results}

\section{Discussion}

\section{Data Availibility}



\acknowledgments

I would like to thank.



This work made use of the \tess\ catalog on ExoFOP

\vspace{5mm}
\facilities{\tess, \gaia, \kepler, etc.}

\software{astropy \citep{Astropy-Collaboration:2013,Astropy-Collaboration:2018}}

\appendix

\section{An Appendix}

Words.

\bibliography{atlas}{}
\bibliographystyle{aasjournal}

\end{document}
